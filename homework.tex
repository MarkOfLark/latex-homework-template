\documentclass{article}

\usepackage{fancyhdr}
\usepackage{extramarks}
\usepackage{amsmath}
\usepackage{amssymb}
\usepackage{amsthm}
\usepackage{amsfonts}
\usepackage{tikz}
\usepackage[plain]{algorithm}
\usepackage{algpseudocode}
\usepackage{lastpage}
\usepackage{caption}
\usepackage{subcaption}
\usetikzlibrary{automata,positioning}

%
% Basic Document Settings
%

\topmargin=-0.45in
\evensidemargin=0in
\oddsidemargin=0in
\textwidth=6.5in
\textheight=9.0in
\headsep=0.25in

\linespread{1.1}

\pagestyle{fancy}
\lhead{\hmwkAuthorName}
\chead{\hmwkClass\ (\hmwkClassInstructor\hmwkClassTime): \hmwkTitle}
%\rhead{\firstxmark}
\rhead{\hmwkDueDate}
%\lfoot{\lastxmark}
\cfoot{\thepage\ of \pageref{LastPage}}

\renewcommand\headrulewidth{0.4pt}
\renewcommand\footrulewidth{0.4pt}

\setlength\parindent{0pt}

%
% Create Problem Sections
%

\newcommand{\enterProblemHeader}[1]{
    \nobreak\extramarks{}{Problem #1 continued on next page\ldots}\nobreak{}
    \nobreak\extramarks{Problem #1 (continued)}{Problem #1 continued on next page\ldots}\nobreak{}
}

\newcommand{\exitProblemHeader}[1]{
    \nobreak\extramarks{Problem #1 (continued)}{Problem #1 continued on next page\ldots}\nobreak{}
    \nobreak\extramarks{Problem #1}{}\nobreak{}
}

\newcounter{partCounter}
\nobreak\extramarks{Problem 1}{}\nobreak{}

%
% Homework Problem Environment
%
% This environment takes an optional argument. When given, it will adjust the
% problem counter. This is useful for when the problems given for your
% assignment aren't sequential. See the last 3 problems of this template for an
% example.
%
\newenvironment{homeworkProblem}[1][1]{
    \setcounter{secnumdepth}{0}
    \section{Problem #1}
    \enterProblemHeader{Problem #1}
}{
    \exitProblemHeader{\thesection}
}

%
% Homework Details
%   - Title
%   - Due date
%   - Due Time
%   - Class
%   - Section/Time
%   - Instructor
%   - Author
%

\newcommand{\hmwkTitle}{Paper Title}
\newcommand{\hmwkDueDate}{Due Date}
\newcommand{\hmwkClass}{Class Title}
\newcommand{\hmwkClassTime}{} % include a leading space if you choose to use this field
\newcommand{\hmwkClassInstructor}{Instructor}
\newcommand{\hmwkAuthorName}{\textbf{Author}}

%
% Title Page
%


\title{
    \vspace{2in}
    \textmd{\textbf{\hmwkClass:\ \hmwkTitle}}\\
    \normalsize\vspace{0.1in}\small{Due\ on\ \hmwkDueDate}\\
    \vspace{0.1in}\large{\textit{\hmwkClassInstructor\hmwkClassTime}}
    \vspace{2in}
}

\author{\hmwkAuthorName}
\date{}

\renewcommand{\part}[1]{\textbf{\large Part \Alph{partCounter}}\stepcounter{partCounter}\\}

%
% Various Helper Commands
%

% Useful for algorithms
\newcommand{\alg}[1]{\textsc{\bfseries \footnotesize #1}}

% For derivatives
\newcommand{\deriv}[1]{\frac{\mathrm{d}}{\mathrm{d}x} (#1)}

% For partial derivatives
\newcommand{\pderiv}[2]{\frac{\partial}{\partial #1} (#2)}

% Integral dx
\newcommand{\dx}{\mathrm{d}x}

% Alias for the Solution section header
\newcommand{\solution}{\textbf{\large Solution}}

% Probability commands: Expectation, Variance, Covariance, Bias
\newcommand{\E}{\mathrm{E}}
\newcommand{\Var}{\mathrm{Var}}
\newcommand{\Cov}{\mathrm{Cov}}
\newcommand{\Bias}{\mathrm{Bias}}

\begin{document}

\maketitle

\pagebreak

\begin{homeworkProblem}[1.5]
	Let G be the event that a student is a genius. \\
	Let C be the event that a student loves chocolate. \\
	Let L be the event that a student is not a genius and does not love chocolate. \\
	
	Given that: \\
	\begin{align}
	P(G) &= 0.6, \\
	P(C) &= 0.7, \\
	P(G \cap C) &= 0.4
	\end{align}
	\\
	And recognizing that \\
	\begin{align}
	L = G^\complement \cap C^\complement
	\end{align}
	\\
	By taking the complement of both sides and applying De Morgan's law we get \\
	\\
	\begin{align}
	L^\complement &= G \cup C
	\end{align}
	\\
	Then recognizing that \(P(L) = 1 - P(L^\complement)\) we get \\
	\\
	\begin{align}
	P(L) &= 1 - P(G \cup C)
	\end{align}
	\\
	Which expands to \\
	\\
	\begin{align}
	P(L) &= 1 - P(G) - P(C) + P(G \cap C)
	\end{align}
	\\
	Filling in the given probabilities from above we get \\
	\\
	\begin{align}
    P(L) &= 1 - 0.6 - 0.7 + 0.4 \\
	&= \mathbf{0.1}
	\end{align}
\end{homeworkProblem}

\pagebreak

\begin{homeworkProblem}[1.6]
	Given that that there is a six-sided die loaded such that an even roll is twice as likely as an odd roll. And given that all even faces are just as likely and that all odd faces are just as likely. We can state the following. \\
	\\
	The sample space is \(\Omega = \{1,2,3,4,5,6\}\) \\
	Let the event of an even roll be \(R_e = \{2,4,6\}\) \\
	Let the event of an odd roll be \(R_o = \{1,3,5\}\) \\
	\\
	First find the probability of each face \\
	\begin{align}
	P(R_e) &= 2P(R_o) \\
	1 &= 2P(R_o) + P(R_o) \\
	1 &= 3P(R_o) \\
	P(R_o) &= \mathbf{\frac{1}{3}} \\
	P(R_e) &= \mathbf{\frac{2}{3}}
	\end{align}
	\\
	Now recognizing that, for a single roll of the die the event of getting a particular face is disjoint from the events of getting any other face, we determine the probabilities for each odd face. \\
	\begin{align}
	P(R_o) &= P(\{1\}) + P(\{3\}) + P(\{5\}) = 3P(\{1\})\\
	\frac{1}{3} &= 3P(\{1\}) \\
	P(\{1\}) + P(\{3\}) + P(\{5\}) &= \frac{1}{9}
	\end{align}
	\\
	Following the same process we determine the probabilities for each even face. \\
	\begin{align}
	P(\{2\}) + P(\{4\}) + P(\{6\}) = \frac{2}{9}
	\end{align}
	\\
	Finally we wish to know for a single roll of the die what is the probability that the outcome is less than four.
	\begin{align}
	P(\{1,2,3\}) &= P(\{1\} \cup \{2\} \cup \{3\}) \\
	& = P(\{1\}) + P(\{2\}) + P(\{3\}) \\
	& = \mathbf{\frac{4}{9}}
	\end{align}
	\\
\end{homeworkProblem}

\pagebreak

\begin{homeworkProblem}[1.7]
	We are asked to find the sample space for when a four sided die is rolled repeatedly, until (if ever) an even number is rolled. \\
	\\	
	Let  \(E\) represent an even roll and \(O\) represent an odd roll. \\
	\\
	The samples space is every sequence that start starts with some number of \(O\)'s (which could be zero) and ends with an \(E\). In addition that those sequences the sample space also contains the sequence of an infinite number \(O\)'s that is not terminated with an \(E\).
	
	\begin{align}
	\Omega = \{\{E\},\{O,E\},\{O,O,E\}, ... \{O,O, ...,E\},\{O,O,...\}\}
	\end{align}
	Where \(E\) is a roll from the set \(\{2,4\}\) and
	Where \(O\) is a roll from the set \(\{1,3\}\).
\end{homeworkProblem}

%\pagebreak

\begin{homeworkProblem}[1.10]

	\def\Ac{A^\complement}
	\def\Bc{B^\complement}
	
    Show that \(P((A \cap \Bc) \cup (\Ac \cap B)) = P(A) + P(B) - 2P(A \cap B)\). \\
    \\
%    Visually this is quite easy to see. When we compute \(P(A) + P(B)\) we end up adding the overlapped region twice. This region is \(P(A \cap B)\). In order to get the desired probability we have to subtract this twice from the original sum.
%    
%    \def\Unirect{(-1.5,-2) rectangle (3,2)}
%    \def\Acircle{(0,0) circle (1cm)}
%	\def\Bcircle{(0:1.5cm) circle (1cm)}
%
%	\def\BeginDiagram{\begin{scope}[shift={(3cm,0cm)}]}
%	\def\EndDiagram{\draw \Unirect node [below left] {$\Omega$};\draw \Acircle node {$A$};\draw \Bcircle node {$B$};\end{scope}}
%	
%	\begin{figure}[!htbp]
%		\begin{subfigure}[b]{0.33\textwidth}
%			\centering
%			\begin{tikzpicture}
%				\BeginDiagram
%				\EndDiagram
%			\end{tikzpicture}
%			\caption{The Universe}
%			\label{fig:universe}
%		\end{subfigure}%
%		\begin{subfigure}[b]{0.33\textwidth}
%			\centering
%			\begin{tikzpicture}
%				\BeginDiagram
%					\begin{scope}[even odd rule]% first circle without the second
%						\clip \Acircle \Unirect;
%						\fill[yellow] \Unirect;
%					\end{scope}
%				\EndDiagram
%			\end{tikzpicture}
%			\caption{\(\Ac\)}
%			\label{fig:Acomp}
%        \end{subfigure}%
%        \begin{subfigure}[b]{0.33\textwidth}
%			\centering
%			\begin{tikzpicture}
%				\BeginDiagram
%					\begin{scope}[even odd rule]% first circle without the second
%						\clip \Bcircle \Unirect;
%						\fill[yellow] \Unirect;
%					\end{scope}
%				\EndDiagram
%			\end{tikzpicture}
%			\caption{\(\Bc\)}
%			\label{fig:Bcomp}
%        \end{subfigure}%
%        
%        \bigskip
%        
%        \begin{subfigure}[b]{0.33\textwidth}
%			\centering
%        		\begin{tikzpicture}
%				\BeginDiagram
%					\begin{scope}[even odd rule]
%						\clip \Bcircle \Unirect;
%						\fill[yellow] \Acircle;
%					\end{scope}
%				\EndDiagram
%			\end{tikzpicture}
%			\caption{\(A \cap \Bc\)}
%			\label{fig:AintBcomp}
%        \end{subfigure}%
%        \begin{subfigure}[b]{0.33\textwidth}
%			\centering
%        		\begin{tikzpicture}
%				\BeginDiagram
%					\begin{scope}[even odd rule]
%						\clip \Acircle \Unirect;
%						\fill[yellow] \Bcircle;
%					\end{scope}
%				\EndDiagram
%			\end{tikzpicture}
%			\caption{\(\Ac \cap B\)}
%			\label{fig:AcompintB}
%        \end{subfigure}%
%        \begin{subfigure}[b]{0.33\textwidth}
%			\centering
%        		\begin{tikzpicture}
%				\BeginDiagram
%					\begin{scope}[even odd rule]
%						\clip \Bcircle \Unirect;
%						\fill[yellow] \Acircle;
%					\end{scope}
%					\begin{scope}[even odd rule]
%						\clip \Acircle \Unirect;
%						\fill[yellow] \Bcircle;
%					\end{scope}
%				\EndDiagram
%			\end{tikzpicture}
%			\caption{\((A \cap \Bc) \cup (\Ac \cap B)\)}
%			\label{fig:wholeexp}
%        \end{subfigure}%
%        \caption{Venn diagrams for problem 1.10}\label{fig:problem1_10}
%	\end{figure}
	
	We can show this using set-theoretic operations.
	\small
	\begin{align}
	P((A \cap \Bc) \cup (\Ac \cap B))
	P((A \cap \Bc) \cup \Ac) \cap ((A \cap \Bc) \cup B) && \text{Distributing first term through} \\
	P(((\Ac \cup A) \cap (\Ac \cup \Bc)) \cap ((A \cup B) \cap (\Bc \cup B))) && \text{Distributing again} \\
	P((\Omega \cap (\Ac \cup \Bc)) \cap ((A \cup B) \cap \Omega)) && \text{\(\Omega = S \cup S^\complement\)} \\
	P((\Ac \cup \Bc) \cap (A \cup B)) && \text{\(\Omega \cap S = S\)} \\
	1-P((\Ac \cup \Bc)^\complement \cup (A \cup B)^\complement) && \text{De Morgan's Law, and \(1-P(S) = P(S^\complement)\)} \\
	1-P((A \cap B) \cup (\Ac \cap \Bc)) && \text{Applying De Morgan's law to interior terms} \\
	-P(A \cap B) + 1 - P(\Ac \cap \Bc) + P((A \cap B) \cap (\Ac \cap \Bc)) && \text{\(P(X \cup Y) = P(X)+P(Y)-P(Y \cap Y)\)} \\
	-P(A \cap B) + P(A \cup B) + P((A \cap B) \cap (\Ac \cap \Bc)) && \text{De Morgan's Law, and \(1-P(S) = P(S^\complement)\)} \\
	-P(A \cap B) + P(A \cup B) + 1 - P((\Ac \cup \Bc) \cup (A \cup B)) && \text{De Morgan's Law, and \(1-P(S) = P(S^\complement)\)} \\
	-P(A \cap B) + P(A \cup B) + 1 - P(\Omega) && \text{\(\Omega = S \cup S^\complement\)} \\
	-P(A \cap B) + P(A \cup B) && \text{\(P(\Omega) = 1\)} \\
	-P(A \cap B) + P(A) + P(B) - P(A \cap B) && \text{\(P(X \cup Y) = P(X)+P(Y)-P(Y \cap Y)\)} \\
	P(A) + P(B) - 2P(A \cap B) && \text{QED}
	\end{align}
	\normalsize
	

\end{homeworkProblem}

\pagebreak

\end{document}
